\documentclass[11pt]{article} 

\begin{document}

\tableofcontents

\title{My Fifth \LaTeX \ Tutorial document}
\author{Hussein Faara}
\date{\today}
\maketitle

\begin{center}\textbf{Fonts}:\end{center}

\begin{center}
This will produce \textit{italicized} text

This will produce \textbf{bold-face} text

This will produce \textsc{small caps} text

This will produce \texttt{typewriter} font

Eg:
Please  visit mrs. Krummel's website at \texttt{http://mrskrummel.com}.

Please excuse my dear aunt Sally.

Please excuse my \begin{large}dear aunt Sally \end{large}

Please excuse my \begin{Large}dear aunt Sally \end{Large}

Please excuse my \begin{huge}
aunt Sally \end{huge}

Please excuse my \begin{Huge}
aunt Sally \end{Huge}

Please excuse my \begin{small}
aunt Sally \end{small}

Please excuse my \begin{tiny}
aunt Sally \end{tiny}

\end{center}

\begin{center}\textbf{Justification}:\end{center}

\begin{center}
This is a centered text.
\end{center}

\begin{flushleft}
This is left-justified
\end{flushleft}

\begin{flushright}
This is right-justified
\end{flushright}


\section{Linear Functions}
	\subsection{Slope-Intercept Form}
	The slope-intercept form of a linear function is given by $y=mx + c $
	\subsection{Standard Form}
	\subsection{Point-Slope Form}
	

\section{Quadratic functions}
	\subsection{Vertex Form}
	\subsection{Standard Form}
	\subsection{Factor Form}

\section*{Linear Functions}
	\subsection*{Slope-Intercept Form}
	The slope-intercept form of a linear function is given by $y=mx + c $
	\subsection*{Standard Form}
	\subsection*{Point-Slope Form}
	
\section*{Quadratic functions}
	\subsection*{Vertex Form}
	\subsection*{Standard Form}
	\subsection*{Factor Form}

\end{document}